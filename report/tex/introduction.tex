\chapter*{Introduction}

Weather prediction is a challenging task for computers. A large number of factors, drawn from large data sets, interact in ways governed by complex physical equations. In order to solve the governing equations of the atmosphere, the computations are discretized on grids spanning the globe (or some local segment of it). Methods such as the \emph{finite differences} and \emph{finite volume} method solve these equations by performing (simple) calculations on each cell that depend only on a limited neighborhood of that cell. This type of computation, called a \emph{stencil}, can be implemented efficiently on graphics processing units, which provide a massively parallel architecture. 

Needless to say, a weather prediction computation for some instant in the future is only useful if the computation terminates before that point in time. Therefore, performance is a critical aspect of any application in weather prediction. The desire for more accurate results, on the other hand, opposes fast runtimes.

More fine-grained grids provide more accurate results but naturally lead to higher data traffic and slower runtimes. So-called \emph{unstructured grids} provide a compromise between globally coarse and globally fine-grained regular grids. Unstructured grids enable higher resolutions in areas of interest while other areas that require less detail can be covered by larger cells. Because of their inherent irregularities, however, unstructured grids add additional overhead. Finding the location of neighboring cells' values especially becomes more involved, requiring additional memory lookups.

The main aim of this thesis is to contrast stencil performance in regular and unstructured grids, as well as exploring some means of optimization. The discrete approximations of the atmosphere's equations are of low arithmetic intensity -- the computations performed on each cell are simple. However, the operations performed are very memory-bandwidth hungry, as each cell requires a lot of input data (several neighboring cells). Thus data locality greatly aids performance. The major focus of this report thus lies in evaluating different memory access strategies and schemes for optimizing the neighborship lookups in unstructured grids for three selected stencils.

In the following sections, we first explain the characteristics of grids and stencils in more detail and elaborate on the architecture of GPUs, on which these stencil calculations are performed. We continue by showing our pursued methods of implementing regular and unstructured grid memory layouts and show different unoptimized and optimized means of accessing grid elements in stencil computations from the GPU. In section \ref{sec:results}, we finally present the observed overhead that the use of unstructured grids imposed when calculating identical stencils on identical data.