\chapter{Grid Access Strategies} \label{sec:optimizations}

In a stencil, computation of the output value of one cell requires access to a neighborhood of cells. Accessing a cell's neighbor's value entails determining the memory index of the desired neighbor. For structured grids, this task involves only arithmetic. For unstructured grids, neighbor access in the X-Y-plane requires an additional memory lookup. This is not necessary for neighbors in the Z-direction, as the grid is regular in this direction.

In this chapter, we describe how stencils can obtain the index of required neighbors and access the grid in optimized ways.

\section{Naive Grid Access and Index Variables}

\subsection{Naive} In the \emph{naive} grid access approach, one thread is mapped to each output cell (total of $d_xd_yd_z$ threads). The indexing and neighborship calculations (including the memory lookup required for unstructured grids) are (re-)performed each time a cell's value is accessed in a stencil.  One inefficiency of this approach is that most stencils require the same neighbors multiple times in different parts of their calculations. Even though the structure of the grid does not change, the indexing calculations are redone in the naive approach on every neighbor access.

\subsection{Index Variables} The \emph{index variables} approach (\emph{idxvar} for short) addresses the issue described for the \emph{naive} approach. There is also one thread per output cell. In this variant, in the first phase of the kernel, all required neighboring cell's indices are determined and stored in variables. These index variables are then used whenever a neighboring cell's value needs to be accessed. This ensures that indexing/neighborship operations (including lookups) are only performed once, even if the same cell is accessed multiple times within one thread. The additional index variables potentially increase the register usage of the kernel, but they reduce the number of expensive memory lookups into the neighborship table if the same neighbors are re-accessed within the same kernel.

\section{Optimizations Making Use of the Z-Regularity}

\subsection{Index Variables + Shared Memory} In this approach (\emph{shared} for short), the stencil is implemented such that there is one thread per output cell (as in \emph{naive} and \emph{idxvar}). However, not all threads perform the neighbor index lookups. Instead, for each cell in the X-Y-plane, one designated ``leader'' thread in each block performs the index calculation for all required neighbors (at the $Z=0$ level). A thread is a leader if the Z-index modulo the Z-block-size is zero, i.e. it has the lowest Z-coordinate of that block. After the lookup, the designated leaders store the required neighbor indices in shared memory.  At this point, all threads in the block synchronize. All threads then access shared memory to obtain the required indices. They add the appropriate constant Z-stride to obtain the index of the neighbors at their respective Z-level. Using this approach, the regularity of the grid in the Z-dimension is exploited; only one global memory lookup for the neighborship information per block is performed. The shared memory lookups are cheaper than lookups in global memory, but synchronization adds some overhead. For this approach to be effective, the block size in the Z dimension needs to be large enough, and certainly larger than one (otherwise, all threads are leaders).

\subsubsection{Bank Conflicts}

As mentioned in section \ref{sec:bank-conflicts}, bank conflicts occur in the Volta architecture when two threads try accessing addresses that are equal modulo $32$. Bank conflicts have to be avoided in order to make use of the full performance of shared memory.

A thread in the \emph{shared} access strategy stores or reads multiple shared memory slots (one for each neighbor the stencil requires). Assume we store neighborship information of each cell contiguously in shared memory. If the total number of required neighbors is very unfortunate, for example $8$ neighbors per cell (equals an array of $32$ bytes), then all threads end up (trying to) access the same memory bank simultaneously in the index lookup phase. This happens because the length of the neighborship array happens to align all top neighbors of cells into one bank, all left neighbors into another bank, etc. In the index lookup phase, all threads then attempt to load their top (left, ...) neighbor pointer at the same time.

In the \emph{shared} access strategy we address this problem by adding padding to the shared memory storage of neighbors. Thus, neighbors are not stored contiguously, but have gaps in between them. The padding is chosen as the smallest value such that the padded array length is coprime with $32$. This guarantees the least possible number of bank conflicts; pointers to the same type of neighbor are spread out evenly across banks for different cells. In other words, all top neighbors are spread evenly across banks, all left neighbors are spread evenly across banks (but may coincide banks with top neighbors, because those are not accessed at the same time instant), etc.

\subsubsection{Warp Broadcasting}

\begin{table}
	\begin{center}
    \begin{tabular}{l l l l}
        \hline
        \textbf{Block size} & \textbf{Variant} & \textbf{Runtime} \\
        \hline
        \hline
        $128\times 1 \times 8$ & Shared memory        & $\mathbf{684 \mu s}$ \\
                               & Warp broadcasting & $737 \mu s$ \\
        \hline
        $32\times 1\times 8$ & Shared memory & $\mathbf{675 \mu s}$ \\
                             & Warp broadcasting & $693 \mu s$ \\
        \hline
        $1\times 1\times 32$ & Shared memory & $13654 \mu s$ \\
                             & Warp broadcasting & $\mathbf{12711 \mu s}$ \\
        \hline
    \end{tabular}
	\end{center}
    \caption{\label{tab:warp-broadcasting}Median runtimes (20 runs) for the Laplace-of-Laplace benchmark (z-curves memory layout, pointer chasing, uncompressed) for three select block sizes. We observe that warp broadcasting is only faster than shared memory in the last block size configuration, which is the slowest overall. We therefore did not further investigate the use of warp broadcasting.}
\end{table}
% ./gridbenchmark --no-verify --size 512x512x64 --threads 128x1x8 32x1x1 32x1x8 8x1x32 1x1x32 laplap-unstr-idxvar-shared -z laplap-unstr-idxvar-warp-shared -z 
% Benchmark                           , Precision, Domain size,,, Blocks     ,,, Threads    ,,, Kernel-only execution time                
%                                    ,          ,   X,   Y,   Z,   X,   Y,   Z,   X,   Y,   Z,   Average,    Median,   Minimum,   Maximum
% laplap-unstr-idxvar-shared-z-curves-4,    double, 512, 512,  64,   4, 508,   8, 128,   1,   8,    685276,    684060,    675086,    696476
% laplap-unstr-idxvar-shared-z-curves-4,    double, 512, 512,  64,  16, 508,  64,  32,   1,   1,   1038436,   1037644,   1027729,   1046049
% laplap-unstr-idxvar-shared-z-curves-4,    double, 512, 512,  64,  16, 508,   8,  32,   1,   8,    673723,    674726,    654446,    691286
% laplap-unstr-idxvar-shared-z-curves-4,    double, 512, 512,  64,  64, 508,   2,   8,   1,  32,   2114038,   2114163,   2111828,   2116009
% laplap-unstr-idxvar-shared-z-curves-4,    double, 512, 512,  64, 508, 508,   2,   1,   1,  32,  13804636,  13653834,  13220255,  14550507
% laplap-unstr-idxvar-warp-shared-z-curves-4,    double, 512, 512,  64,   4, 508,   8, 128,   1,   8,    737049,    737926,    725546,    742856
% laplap-unstr-idxvar-warp-shared-z-curves-4,    double, 512, 512,  64,  16, 508,  64,  32,   1,   1,   1010478,   1010224,   1004369,   1017049
% laplap-unstr-idxvar-warp-shared-z-curves-4,    double, 512, 512,  64,  16, 508,   8,  32,   1,   8,    695580,    693476,    683075,    766056
% laplap-unstr-idxvar-warp-shared-z-curves-4,    double, 512, 512,  64,  64, 508,   2,   8,   1,  32,   1860175,   1860061,   1856796,   1863697
% laplap-unstr-idxvar-warp-shared-z-curves-4,    double, 512, 512,  64, 508, 508,   2,   1,   1,  32,  12711574,  12712320,  12702761,  12722870

Loads and stores to shared memory produce some overhead. A more lightweight alternative provided by the \emph{CUDA} model are so-called \emph{warp-level primitives}. Those allow threads that are \emph{within the same warp} to exchange data more efficiently -- in a direct register-to-register fashion.

We briefly experimented with a variant of \emph{shared} strategy, using warp broadcasting instead of shared memory. In this variant, all threads in the first lane (first thread of each warp) are the designated leader threads that do the actual neighbor lookup. Other threads in the same warp receive the leader thread's neighbor indices by use of the \texttt{\_\_shfl\_sync()} method.

While the actual warp broadcasting is more lightweight than shared memory, it is also more limited: only threads within the same warp can exchange data. This sets restrictions on the kernel launch configuration; the threads in a block have to be allocated such that cells across different Z-levels fall within the same warp (otherwise no neighbor pointer sharing can take place). To make use of warp broadcasting, some additional calculations also need to be made (determining the lane ID, masks of active threads). For those reasons, warp broadcasting appeared to be slightly slower than the shared memory access variants in most realistic scenarios. In launch configurations with many threads in the X- or Y-dimension, warp broadcasting even was considerably slower (due to no sharing being possible within warp limits anymore). However, in launch configurations with many threads in the Z-dimension, warp broadcasting slightly outperformed shared memory. These launch configurations are rather slow overall (compared to other block sizes), though, and are therefore not very useful. We thus did not further investigate warp broadcasting use.

See table \ref{tab:warp-broadcasting} for a comparison of the warp broadcasting approach to using shared memory on an exemplary benchmark.

\subsection{Index Variables + Z-loop} In this approach (\emph{z-loop} for short), there are only $d_x\cdot d_y$ threads, i.e. only one thread per pillar of cells with equal X- and Y-coordinates. In a loop over all Z-levels, a thread calculates the results for all cells with equal X- and Y-coordinates (hence the name). The indices of all required cells at the $z=0$ level are stored in \emph{index variables} before the start of the loop. The regularity of the grid in Z-direction enables us to update the index variables in each iteration of the loop by simply adding a Z-stride. There is no memory lookup into the neighborship table inside the loop. Thus, even in the unstructured grid case, neighborship table lookups are only necessary once before the loop. This comes at the cost of possibly reduced occupancy, however, as there is a lower total number of threads.

\subsection{Index Variables + Sliced Z-loop} This variant addresses the issue of low occupancy in the above approach. It is practically identical to it, but splits the Z-loop up in smaller chunks. There are $d_x\cdot d_y\cdot \frac{d_z}{m}$ threads, where $m$ is the number of output cells in the Z-direction a single thread should calculate. In the following benchmarks, we used $m=8$.

%\paragraph{Loop Unrolling}
% TODO mention effects of loop unrolling in the latter two variants

