\section{Grid Access Strategies} \label{sec:optimizations}

In a stencil, computation of the output value of one cell requires access to a neighborhood of cells. Accessing a cell's neighbor's value entails determining the coordinates of the desired neighbor and subsequently calculating the memory index of those coordinates. For structured grids, both of those tasks involve only arithmetic (as described in the indexing section). For unstructured grids, neighbor access in the X-Y-plane requires an additional memory lookup. This is not necessary for neighbors in the Z-direction, as the grid is regular in this direction.

In this section, we describe how stencils can obtain the index of required neighbors and access the grid in optimized ways.

\subsection{Naive Grid Access and Index Temporaries}

\paragraph{Naive} In the \emph{naive} grid access approach, one thread is mapped to each output cell (total of $d_x*d_y*d_z$ threads). The indexing and neighborship calculations (including the memory lookup required for unstrucutred grids) are (re-)performed each time a cell's value is accessed in a stencil.  One inefficiency of this approach is that most stencils require the same neighboring cells multiple times in different parts of their calculations. Even though the structure of the grid does not change, the indexing calculations are redone in the naive approach on every neighbor access.

\paragraph{Index Temporaries} The \emph{index temporaries} (\emph{idxvar} for short) approach (\textbf{idvar} for short) addresses issue described for the \emph{naive} approach. There is also one thread per output cell. In this variant, in a first phase of the kernel, all required cell's indices are calculated and stored in variables. These index variables are then used whenever a neighboring cells value needs to be accessed. This ensures that indexing/neighborship calculations are only performed once, even if the same cell is accessed multiple times within one thread. The additional index variables potentially increase the register usage of the kernel, but they reduce the amount of expensive memory lookups into the neighborship table if the same neighbors are re-accessed within the same kernel.

\subsection{Optimizations Making Use of the Z-Regularity}

\paragraph{Index Temporaries + Shared Memory} In this approach (\emph{idxvar-shared} for short), the stencil is implemented such that there is one thread per output cell (as in \emph{naive} and \emph{idxvar}). However, not all threads perform the neighbor index lookups. Instead, for each cell in the X-Y-plane, designated threads in each block perform the index calculation/lookup of all required neighboring cells at the $Z=0$ level. The designated threads store the result of that computation/lookup in shared memory. If the Z-index modulo the Z-block-size is zero, a thread is a designated lookup-thread. After the neighboring cells indices at the lowest Z-level are determined and stored in shared memory, all threads in the block synchronize. All threads then access shared memory to obtain the required indices. They add the appropriate constant Z-stride to them to obtain the index of the neighbors at their Z-level. Using this approach, the regularity of the grid in the Z-dimension is exploited in order to only perform one global memory lookup per block for the neighborship information. The shared memory lookups are cheaper than lookups in global memory, however synchronization adds some overhead. For this approach to be effective, the block size in the Z dimension needs to be large enough.

\begin{verbatim}
% TODO
% - detail bank conflict experiments
% - mention that we also tried warp broadcasting
\end{verbatim}

\paragraph{Index Temporaries + K-loop} In this approach (\emph{idxvar-kloop} for short), there are only $d_x\times d_y$ threads, i.e. only one thread per stack of cells in the X-Y-plane. Each thread calculates the results for all cells with equal X- and Y-coordinates in a loop over all Z. The indices of all required cells at the $z=0$ level are stored in \emph{index variables} before the start of the loop. Regularity of the grid in Z-direction enables us to update the index variables in each iteration of the loop by simply adding a Z-stride. There is no memory lookup into the neighborship table inside the loop. Thus, even in the unstructured grid case, memory lookups are only necessary once before the loop. This comes at the cost of possibly reduced occupancy, however, as there are fewer threads.

\paragraph{Index Temporaries + Sliced K-loop} This variant addresses the issue of low occupancy in the above approach. It is practically identical to it, but splits the Z-loop up in smaller chunks. There are $d_x\times d_y\times \frac{d_z}{m}$ threads, where $m$ is the number of output cells in the Z-direction a single thread should calculate. In the following benchmarks, we used $m=8$.

%\paragraph{Loop Unrolling}
% TODO mention effects of loop unrolling in latter two variants

