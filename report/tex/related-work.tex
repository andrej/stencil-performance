\chapter*{Related Work}

Solano-Quinde et al. \cite{solano2011} present an algorithm for implementing scientific analyses in unstructured grids on GPUs for more general problems. They identify occupancy and memory access as the main limiting factors of performance. Their paper also explores the implications of using different memory layouts for the unstructured grid representation, arriving at the conclusion that struct-of-array-type layouts (see also section \ref{sec:representing-multiple-fields} in this report) are better suited for GPUs because of coalescing concerns. 

In this report, we assume the unstructured grid on top of which to apply a stencil is already given. An overview of the various methods of how such grids can be generated is given in \cite{mavriplis1997}. Section 3.4 of the publication also briefly hints at how an unstructured grid may be stored. This forms the basis of our initial naive storage approach.

A lot of research has been carried out concerning the compressed storage of meshes \cite{edgebreaker}\cite{edgebreaker-quadrilateral}. However, most of those approaches are not applicable to our problem, where decompression has to be virtually free and reordering of values is not permittable. The approach presented for compressing social network graphs in \cite{social} is similar to our compression approach.